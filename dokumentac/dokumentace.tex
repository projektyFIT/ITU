%%%%%%%%%%%%%%%%%%%%%%%%%%%%%%%%%%%%%%%%%%%%%%%%%%%%%%%%%%%%%%%%%%%%%%%%%%%%%%
%%%%%%%%%%%%%%%%%%%%%%%%%%%%%%%%%%%%%%%%%%%%%%%%%%%%%%%%%%%%%%%%%%%%%%%%%%%%%%
%%

%%%%%%%%%%%%%%%%%%%%%%%%%%%%%%%%%%%%%%%%%%%%%%%%%%%%%%%%%%%%%%%%%%%%%%%%%%%%%%
%%%%%%%%%%%%%%%%%%%%%%%%%%%%%%%%%%%%%%%%%%%%%%%%%%%%%%%%%%%%%%%%%%%%%%%%%%%%%%
\documentclass[12pt,a4paper,titlepage,final]{article}
\newcommand{\uv}[1]{\quotedblbase #1\textquotedblleft}
% cestina a fonty
\usepackage[czech]{babel}
\usepackage[utf8]{inputenc}
% balicky pro odkazy
\usepackage[bookmarksopen,colorlinks,plainpages=false,urlcolor=blue,unicode]{hyperref}
\usepackage{url}
% obrazky
\usepackage[dvipdf]{graphicx}
% velikost stranky
\usepackage{graphicx}
\usepackage[top=3.5cm, left=2.5cm, text={17cm, 24cm}, ignorefoot]{geometry}

\begin{document}

\begin{titlepage}

\begin{figure}[h]
\begin{center}
\includegraphics[scale=0.6]{logo.eps}
\end{center}
\end{figure}

\begin{center}
\LARGE
\textsc{Vysoké učení
  technické v~Brně\\ \Large{Fakulta informačních technológií}}\\
\vspace{\stretch{0.382}}
\LARGE
Elektronická kuchařka \\
\Huge
Dokumentácia z predmetu ITU\\ 
\large{\medskip
\today }\\
\vspace{\stretch{0.618}}
\end{center}
 \hfill   

\begin{flushleft}
\begin{large}
\begin{tabular}{ll}
\textbf{Varianta - č.22} \\ 


 \\Vojtěch Meca   xmecav00 , \ \\ Jiří Macků  xmacku03 \\ Martin Maga xmagam00\\
Rozšírenie:


\end{tabular}
\end{large}
\end{flushleft}
\end{titlepage}


\tableofcontents
\newpage

\section{Úvod}
Táto dokumentácia sa zaoberá vývojom, implementáciou a testovaním elektornickej kuchařky. Je rozdelená do kapitól, ktoré sa zaoberajú rôznymi aspektami projektu.\cite{Prokop:Algoritmy}.Túto tému sme si vybrali z dôvodu vysokého dopytu užívateľov pre daný typ aplikácie a možnosť súkromného využitia. 


\section{Motivácia}
Počítače stále viac a viac vnikajú do bežných činností človeka a snažia sa mu uľahčiť a urýchliť prácu. V poslednej dobe viac a viac užívateľov má prístup k internetu a využíva ho na získanie všemožných informácií. Jednou s týchto informácií býva často kuchársky recept. Tento recept by mal spĺňať požadované kvality, byť jednoduchý rýchly, overený a jasne formulované. Užívateľ pristupujúcu k takejto služby má istej dobe snahy vložiť si do zbierky vlastné recepty. Tieto recepty chce hodnotiť, alebo ich upravovať po prípade ich zmazať. Jednotlivé recepty by si rád zaradil do kategórií. Z tohto dôvodu sme sa rozhdoli implementovať zadanie Elektronickej kuchařky, ktoré spĺňa všetky tieto požiadavky užívateľov. Snahou bolo zamerať sa na mladých užívateľov, ktorým nerobí problém interakcia s novou aplikáciou, muži alebo ženy. Aplikácia nie je určená pre zrakovo znevýhodneých užívateľov, keďźe aplikácia neobsahuje možnosti pre čítačky rovnako ani aj pre užívateľov staršieho veku, ktorí majú problémy so zrakom.

\section{Grafické rozhranie}
Užívateľské rozhranie bolo navrhované s cieľom maximálne zjednodušiť prácu užívateľa s aplikáciou. Aby kĺadlo dôraz na jednoduchosť a miernu inovatívnosť riešenia. Užívateľ je schopný sa s rozhraním pomerne rýchlo zoznámiť a plne ho využívať pre dosiahnutie svojich výsledkov. Jednotlivé časti sú farebne rozdelené, aby možnosti, ktoré užívateľ bude využívať pomerne často mal stále na očiach. Rozhranie sa skladá z hlavnej lišty, ktorá je prístupná v každom okamihu práce s aplikáciou. Pred samotnou fázou implementácie prebiehali viaceré stretnutia členov tímu, ktorých cieľom bolo návrh optimálne a dostatočne jednoduché rozhranie pre používanie boli tam pridané inovatívne prvky, ktorých cieľom bolo spestriť a zaujať užívateĺa o naše riešenie.\cite{Stephen:Rozhrania}


Výsledkom jednotlivých Obsahuje možnosti na zobrazenie jednotlivých kategórií jedál, fulltextové vyhľadávanie v názvoch, možnosť pridania receptu  a tak isto možnosť pokročilého vyhľadávania. V ďalšej kapitole sa bude zaoberať implementáciou riešenia. 

\subsection{Inovatívnosť}
Aplikácia neprínašla veľa nových možností. Ponúkala štandardné funkcie, na ktoré sa kladol dôraz, aby boli jednoduché a ľahko používateľné. S inovatívnym prístupom by som rád spomenul možnosti pokročilého vyhľadánia. Užívateľ mal širokú možnosť vyhľadávania a to podľa názvu v recepte, postupu, času prípravy, kategórií receptu.



\section{Implementácia}
Táto kapitola sa zaoberá implementáciou a ukazuje možnosti implementácie elektronickej kuchařky. Pri implementácií bol využitý jazyk HTML verzie 5\footnote{$http://cs.wikipedia.org/wiki/HTML5$}, jazyk CSS verzie3\footnote{$http://www.w3schools.com/css/css3_intro.asp$}, JavaScript\footnote{$http://cs.wikipedia.org/wiki/JavaScript$} a PHP verzie 5\footnote{$http://php.net/$}. Práca bola rozdelená do menších celkov, ktoré boli postupne testované. Jednotlivý členovia tímu pracovali s každou častou aplikácie, minimálne pripomienkovali jeho implementáciu s cieľom vytvoriť aplikáciu, ktorá by bola čo maximálne responzívna. Boli rozvrhnuté činnosti, ktoré by mala aplikácia spĺňať a na základe nich a návrhu užívateľské rozhrania prebiehala samostná implementácia.


\section{Testovanie a vyhodnotenie}
Aplikácia bola umiestnená na školský server odkiaľ ju mali prístupnú všetci užívatelia. Testovanie prebiehalo na základe dotazníka, ktorý bol predloženým cieľovým skupinám. Ďalej prebehlo vyhotovenie testov, kde sa merali rôzne veličiny typicky rýchlosť splnenia danej akcie a počet zbytočných pohybov. Na základe týchto testov sme mohli vyhodniť rôzne aspekty nášho užívatelského rozhrania. 

Na testovanie sme použili nasledujúce testovacie skupiny:
\begin{itemize}
\item Užívatelia od 15 - 25, ktorí majú snahu maximálne využívať informačné technológie
\item Užívatelia 25 - 40, ktorí využívajú informačné technológie v menšom rozsahu
\item Ženy vo veku 20 - 35, ktoré majú skúsenost s informačnými technológiami
\item Užívatelia ženkého pohlavia od 15 - 40, ktorí majú minimálne alebo žiadne skúsenosti s inforačnými technológiami, ale zaujímajú sa o daný typ aplikácie
\end{itemize}
Každá testovacia skupina mala zastúpenie 10 užívateľov.

\subsection{Dotazník}
Dotazník sa skladal z nasledujúcich otázok:
\begin{itemize}
\item Máte skúsenosti s podobnými elektronickými kuchárkami ?
\item Ako hodnotíte užitočnosť takéhoto programu ?
\item Ako sa Vám program ovláda ?
\item Čo Vám v našom program chýba a malo by tam určite byť ?
\item Je niečo čo na našej aplikácií oceňujete ?
\item Máte nejaké iné propomienky k aplikácií ?
\item Vedeli by te používať elektronickú kuchárku miesto jej papierovej podoby ?

\end{itemize}

\subsection{Testy}
Testy boli vyhotované s cieľom zistiť, aké je užívateľské rozhranie jednoduché, intuitívne a ako rýchlo sa dokáže užívateľ zorientovať v aplikácií. 

\subsubsection{Test č.1}
Cieľom tohto testu boli otestovanie možnosti jednoduchého fulltextového vyhľadávania. Užívateľom s cieľom skupiny bola zadaná nasledujúca úloha: "Vyhľadajte jedlá, ktoré obsahujú v jedle "bramb" pomocou fulltextového vyhľadávania". Zapíšte koľko výsledkov vyhľadávanie vrátilo. Užívateľ musí správne lokalizovať pole pre fulltextové vyhľadávanie a zapísať d tohto poľa "bramb". Výsledok vyhľadávania by mal vrátiť 2 položky.

\subsubsection{Test č.2}
Cieľom tohto testu bolo otestovanie pridávania nového receptu. Zadanie znelo "Pridajte nový recept s názvom "Bramborové těstoviny" s dĺžkou prípravy 6 minút, ktorý bude mať 3 ľubovolné suroviny, nahrajte ľubovolný obrázok receptu a napíšte ľubolný recept, ktorý môže obsahovať aj nezmyselné vety." Užívatelia museli správne lokalizovať tlačidlo pre pridanie nového receptu.Následne sa im rozbrazil formulár, do ktorého mohli ľubovoľne písať. Museli správne vyuplniť názov receptu, vybrať zo zoznamu pre dĺžku prípravy interval 5-10 minút. Následne vyplniť 3 suroviny vrátane množstvo. Nahrať obrázok z disku počítača a následne vyplniť návod na prípravu.

\subsubsection{Test č.3}
Cieľom bolo testu bolo otestovať možnosť editovania receptu. Zadanie úlohy znelo nasledovne:"Editujte už pripravený recept. Zmeňte názov na ľubovolný iný, tak isto dĺžku prípravy, zrušte ľubovolnú surovinu a editujte postup". Užívateľ už mal predpripravené okno s editáciou, musel správne zmeniť jednotlivé hodnoty, lokalizovať tlačidlo na zmazanie suroviny a potvrdiť zmenu stlačením tlačidla uložiť.

\subsubsection{Test č.4}
Cieľom toho testu bolo otestovať pokročilé vyhľadávanie. Zadanie úlohy znelo: "Využite možnosť pokročilého vyhľadávania a vyhľadajte všetky recepty s dobou prípravy medzi 5-10 minút". Užívateľ musel správne lokalizovať možnosť pokročilého vyhľadávania. Následne bol presmerovaný na formulár, do ktorého mohol zadať viaceré možnosti. Musel správne vybrať z rolovacieho zoznamu 5-10 minút a nechať ostatné možnosti nezmenené.

\subsubsection{Test č.5}
Cieľom tohto bolo otestovať možnosť hodnotenia jednotlivých receptov. Zadanie úlohy znelo:"Využite možnosť hodnotenia receptu a zvoľte najlepšie hodnotenie pre recept." Užívateľ mal už predpripravené okno s receptom. Užívateľ musel správne lokalizovať možnosť hodnotenia, ktorá bola reprezentovaná hviezdičkami. Musel správne určiť váhu, ktorú musel priradiť receptu. Musel určiť , že 5 hviezdičiek odpovedá najlepšiemu hodnoteniu.




\subsection{Vyhodnotenie dotazníku a testov}

Dotazník bol predložený respondentom, ktorý na to odpovedali v nasledovnej podobe. 


\section{Záver}

\newpage

\section{Referencie}


\bibliographystyle{czechiso}

\bibliography{dokumentace}

\end{document}

