\documentclass[pdf,slideColor,fyma]{prosper}
\usepackage[czech]{babel}
\usepackage[utf8]{inputenc}

\usepackage{color}
\usepackage{graphics}
\usepackage{picture}

\usepackage{url}
%\DefaultTransition{Split}
\slideCaption{IFJ12}

\paperwidth = 597pt
\paperheight = 845pt
\begin{document}
%slide
\title{Obhajoba projektu IFJ12 \\ \medskip  2012/2013}
\subtitle{\medskip Interpreter jazyka IFJ12}
\author{Martin Maga,Vít Mojžíš, Viktor Malík, Vojtěch Meca, Jiří Macků}

\date{10.12.2012}
\maketitle

%slide
\overlays{4}{
\begin{slide}{Rozdelenei projektu}
\begin{itemize}
\item Lexikálny analyzátor
\fromSlide{2}{
\item Syntaktický analyzátor
}
\fromSlide{3}{
\item Sémantický analyzátor
}
\fromSlide{4}{
\item Interpreter
}


\end{itemize}
\end{slide}
}

%slide
\begin{slide}{Lexikálny analyzátor}
\begin{itemize}
\item Konečný automat pre rozpoznanie lexém
\item Identifikátory, čísla(numeric), reťazce, escape sekvencie
\item Rozpoznávanie kľúčových a rezervovaných slov pomocou poľa reťazcov
\item Identifikácia lexikálnych chýb


\end{itemize}
\end{slide}

%slide
\begin{slide}{Syntaxou riadený preklad}

\begin{itemize}
\item Pozostáva: 1. syntaktický analyzátor 2.sémantický analyzátor
\item LL-gramatika, zásobník a postfix notácia
\item Princíp: tabuľka na základe, ktorej je riadená redukcia a vyhodnocovanie výrazov
\item Tabuľka funkcií, ktoré neboli definované pri volaní
\item Ukladanie tokenov do zásobníku a následné vyhodnocovanie
\item Analýza zhora nadol
\item Súčasná syntaktická a sémantická kontrola
\end{itemize}


\end{slide}


%slide
\begin{slide}{Interpreter}
\begin{itemize}
\item Kód inštrukcií: ukazateľ na operand, ukazateľ na funkciu, typ inštrukcie
\item Ukladanie inštrukcií na zásobník
\item Funkcia prog errors interpreter
\item Neznalosť problematiky problému

\end{itemize}

\end{slide}

\begin{slide}{Find}
\begin{itemize}
\item Vstavaná funkcia, ktorá slúži na vyhľadanie podreťazca v reťazci
\item Implementácia algoritmom: \emph{Knuth-Morris-Pratt}
\item Princíp: Konečný automat s 3 uzlami: 1. STOP 2. READ 3. START
\item Tvorba pomocnej tabuľky na porovnávanie
\item Naplnenie tabuľky hodnotami
\item Porovnávanie reťazca s tabuľkou v prípade nezhody pokračuje ďalej
\item Algoritmus končí nájdeným podreťazca a vrátením indexa
\item V prípade neúspechu vracia hodnotu -1
\item Zložitosť: lineárna O(n+m)
\end{itemize}

\end{slide}


\begin{slide}{Sort}
\begin{itemize}
\item Vstavaná funkcia, ktorá slúži k zoradeniu znakov v reťazci
\item Implementácia algoritmom: \emph{Shell-sort}
\item Priama metóda, princip bubble sort-u
\item Princíp: Postupná aplikácia bubble sort-u s rôznymi hodnotami kroku
\item V poslednej fáze je aplikovaný klasický bubble sort
\item Nestabilný,  \uv{in situ}, kvadratická zložitosť
\item Najlepší radiaci algoritmus



\end{itemize}

\end{slide}


\begin{slide}{Tabuľka symbolov}
\begin{itemize}
\item Implementácia hashovacou tabuľkou s jednosmerne viazanými zoznamami synoným
\item Údaj o počte prvkoch v zoznamoch z dôvodu rekurzie
\item Pri vyhľadávaní použitie najnovšej inštancie hľadanej premennej
\item Pri návrate z funkcie su kópie najbližšie vrcholu zásobníku odstránené

\item Zložitosť je daná prístup k nadlhšej postupnosti zreťazených synoným



\end{itemize}

\end{slide}



\begin{slide}{Použité zdroje}
\begin{itemize}
\item \texttt{Opora z Predmetu IAL}
\item \texttt{Opora z Predmetu IFL}
\item \texttt{http://falconpl.org/}
\end{itemize}
\end{slide}

%slide
\begin{slide}{Ďakujem za pozornosť}

\end{slide}


\end{document}
